\documentclass[a4paper, 12pt]{article}
\usepackage{amsmath}
\usepackage{amssymb}
\usepackage{color}
\usepackage{dsfont}
\usepackage[left=1.5cm, right=1.5cm, bottom=2cm, top=2cm]{geometry}
\usepackage{graphicx}
\usepackage[utf8]{inputenc}
\usepackage{microtype}
\usepackage{natbib}
\definecolor{ao(english)}{rgb}{0.0, 0.5, 0.0}

\newcommand{\bjb}{\color{ao(english)}}
\newcommand{\btheta}{\boldsymbol{\theta}}
\newcommand{\given}{\,|\,}

\title{A Trending Algorithm for LBRY Based on Biexponential Kernels}
\author{Brendon J. Brewer}
\date{}

\begin{document}
\maketitle

%\abstract{\noindent For Kevin Knuth.}

% Need this after the abstract
\setlength{\parindent}{0pt}
\setlength{\parskip}{8pt}

Consider an LBC change from $x$ to $x'$ which occurs at time (height) $t_0$.
At the moment, assume positive changes only, i.e., $x' > x$.
Let $\delta$ be the delay, so the kernel peaks at time $t_1 = t_0 + \delta$.
The kernel then decays exponentially. The interval
of time from $t_0$ to $t_1$ is called the `rise', and from $t_1$ onwards it is
the `fall'.

Let the kernel due to this LBC change be of the form
\begin{align}
K(t) &=
    \left\{
        \begin{array}{lr}
            0, & t < t_0 \\
            A\exp\left(2\left(\frac{t - t_1}{\delta}\right)\right),
                                                & t_0 \leq t < t_1 \\
            A\exp\left(\frac{t_1 - t}{\ell}  \right),  & t \geq t_1.
        \end{array}
    \right.
\end{align}
The kernel rises from an initial value of $A\exp(-2) \approx 0.1A$ to a peak
of $A$, and then decays with timescale $\ell$. The kernel parameters
$(A, \delta, \ell)$ can depend on $x$ and $x'$, and potentially other things.

%The reason for making the exponents $(\alpha-1)$ instead of $\alpha$ is
%so I can make analogies to Beta distributions. The normalisation of Beta
%distributions is
%\begin{align}
%\int_0^1 x^{\alpha-1}(1-x)^{\beta-1} \, dx
%    &= \frac{\Gamma(\alpha)\Gamma(\beta)}{\Gamma(\alpha+\beta)}.
%\end{align}
%When $\beta=1$, this reduces to
%\begin{align}
%\int_0^1 x^{\alpha-1}(1-x)^{\beta-1} \, dx
%    &= \frac{\Gamma(\alpha)}{\Gamma(\alpha+1)} \\
%    &= \frac{1}{\alpha}.
%\end{align}


%The integral of the rising part of the kernel is
%\begin{align}
%Z_{\rm rise} &= A\int_{t_0}^{t_0+\delta}
%                    \left(\frac{t - t_0}{\delta}\right)^{\alpha-1} \, dt
%\end{align}
%Letting $\tau = (t-t_0)/\delta$, this becomes
%\begin{align}
%Z_{\rm rise} &= A\delta\int_0^1
%                    \tau^{\alpha-1} \, d\tau \\
%             &= A\delta
%                    \frac{\Gamma(\alpha)\Gamma(1)}{\Gamma(\alpha+1)} \\
%             &= \frac{A\delta}{\alpha}.
%\end{align}
%The integral of the falling part, and therefore the total integral $Z$,
%can be obtained similarly. The total integral is
%\begin{align}
%Z &= \frac{A(\delta + \ell)}{\alpha}
%\end{align}

%Let's quantify the effective log-duration of a kernel by the Shannon entropy of
%its normalised version. First, let's get the Shannon entropy of a subset of
%Beta distributions (those with $\beta=1$):
%\begin{align}
%H
%    &= -\int_0^1 f(x) \log f(x) \, dx \\
%    &= -\int_0^1 \alpha x^{\alpha-1} \log \left(\alpha x^{\alpha-1}\right)
%            \, dx \\
%    &= -\int_0^1 \alpha x^{\alpha-1}
%                \Big[\log\alpha + (\alpha-1)\log x\Big] \, dx \\
%    &= -\log\alpha - \int_0^1 \alpha x^{\alpha-1}(\alpha-1)\log x \, dx.
%\end{align}
%According to {\tt sympy}, this isn't tractable without obscure special
%functions. Oh well.


\end{document}
