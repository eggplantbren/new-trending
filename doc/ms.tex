\documentclass[a4paper, 12pt]{article}
\usepackage{amsmath}
\usepackage{amssymb}
\usepackage{color}
\usepackage{dsfont}
\usepackage[left=1.5cm, right=1.5cm, bottom=2cm, top=2cm]{geometry}
\usepackage{graphicx}
\usepackage[utf8]{inputenc}
\usepackage{microtype}
\usepackage{natbib}
\definecolor{ao(english)}{rgb}{0.0, 0.5, 0.0}

\newcommand{\bjb}{\color{ao(english)}}
\newcommand{\btheta}{\boldsymbol{\theta}}
\newcommand{\given}{\,|\,}

\title{A Trending Algorithm for LBRY Based on Gamma Kernels}
\author{Brendon J. Brewer}
\date{}

\begin{document}
\maketitle

%\abstract{\noindent For Kevin Knuth.}

% Need this after the abstract
\setlength{\parindent}{0pt}
\setlength{\parskip}{8pt}

Consider an LBC change from $x$ to $x'$ which occurs at time (height) $t_0$.
At the moment, assume positive changes only, i.e., $x' > x$.

% var = alpha/beta^2 --> beta = sqrt(alpha)/sigma

Let our gamma kernels be
of the form
\begin{align}
y(t) = A t^{\alpha - 1}
        \exp\left[-t\sqrt{\frac{\alpha}{\sigma^2}}\right].
\end{align}
The peak time, if it exists, is
\begin{align}
t_{\rm peak} &= \sigma\left(\frac{\alpha-1}{\sqrt{\alpha}}\right).
\end{align}
The value of $y(t)$ at its peak is
\begin{align}
y_{\rm peak} &= A t_{\rm peak}^{\alpha - 1}
        \exp\left[-t_{\rm peak}\sqrt{\frac{\alpha}{\sigma^2}}\right]
\end{align}
which can easily be solved for $A$.


\end{document}
